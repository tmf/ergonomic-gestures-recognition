\chapter{Introduction}
\label{chap:introduction}

\section{Context}
\label{sec:context}
In the last two decades, the mouse and the keyboard were the dominant input devices for computer interfaces. Nowadays, new forms of interfaces come up, such as touch interfaces and even touchless interfaces without any additional devices for the user. For simple or short tasks, the pointing finger has been widely accepted, but for prolonged sessions at an computer interface, users still prefer mouses and keyboards. This is due to the \textit{ergonomic features} of these input devices, mainly the arm support and limited action space, but also comfort and precision.\\

%Applications of such an ergonomic gesture recognition range from video collaboration to a completely new way of natural interaction, as seen in the recent Microsoft product Kinect \cite{kinect}.

\section{Ergonomic gestures}
\label{sec:ergonomic-gestures}

A gesture is, according to the definition of Kurtenbach and Hulteen \cite{kurtenbach}, "a motion of the body containing information". There exist several classifications of gestures, but gestural interfaces mainly focus on symbolic and deictic gestures according to the classification of Rim\'{e} and Schiaratura \cite{schiaratura}.

Early attempts of vertical gestural interfaces had a ergonomic problem, later known as the \textit{gorilla arm}. A user is not at ease lifting the arms for a prolonged period, nor is he comfortable at performing specific gestures in a big action space. This project aims at providing a gesture recognition for hand and fingers, in a situation where the forearm is supported at a table or elbow rests in a chair. These conditions limit the action space and focus on more subtle and articulated gestures.

This project will focus on these gestures with ergonomic features for the purpose of controlling an application interface.
\section{Goals}
\label{sec:goal}

Within the context of ergonomic gestures this project aims at:
\begin{itemize}
\item recognize spatial and temporal aspects of confined and small-grained gestures
\item build an architecture capable of \textbf{real-time} gesture recognition
\item implement demonstration program using this architecture
\end{itemize}


